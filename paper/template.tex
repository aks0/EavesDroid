% THIS IS SIGPROC-SP.TEX - VERSION 3.1
% WORKS WITH V3.2SP OF ACM_PROC_ARTICLE-SP.CLS
% APRIL 2009
%
% It is an example file showing how to use the 'acm_proc_article-sp.cls' V3.2SP
% LaTeX2e document class file for Conference Proceedings submissions.
% ----------------------------------------------------------------------------------------------------------------
% This .tex file (and associated .cls V3.2SP) *DOES NOT* produce:
%       1) The Permission Statement
%       2) The Conference (location) Info information
%       3) The Copyright Line with ACM data
%       4) Page numbering
% ---------------------------------------------------------------------------------------------------------------
% It is an example which *does* use the .bib file (from which the .bbl file
% is produced).
% REMEMBER HOWEVER: After having produced the .bbl file,
% and prior to final submission,
% you need to 'insert'  your .bbl file into your source .tex file so as to provide
% ONE 'self-contained' source file.
%
% Questions regarding SIGS should be sent to
% Adrienne Griscti ---> griscti@acm.org
%
% Questions/suggestions regarding the guidelines, .tex and .cls files, etc. to
% Gerald Murray ---> murray@hq.acm.org
%
% For tracking purposes - this is V3.1SP - APRIL 2009

\documentclass{acm_proc_article-sp}

\begin{document}

\title{Keystroke recovery using mobile phone accelerometers}

%
% You need the command \numberofauthors to handle the 'placement
% and alignment' of the authors beneath the title.
%
% For aesthetic reasons, we recommend 'three authors at a time'
% i.e. three 'name/affiliation blocks' be placed beneath the title.
%
% NOTE: You are NOT restricted in how many 'rows' of
% "name/affiliations" may appear. We just ask that you restrict
% the number of 'columns' to three.
%
% Because of the available 'opening page real-estate'
% we ask you to refrain from putting more than six authors
% (two rows with three columns) beneath the article title.
% More than six makes the first-page appear very cluttered indeed.
%
% Use the \alignauthor commands to handle the names
% and affiliations for an 'aesthetic maximum' of six authors.
% Add names, affiliations, addresses for
% the seventh etc. author(s) as the argument for the
% \additionalauthors command.
% These 'additional authors' will be output/set for you
% without further effort on your part as the last section in
% the body of your article BEFORE References or any Appendices.

\numberofauthors{4} %  in this sample file, there are a *total*
% of EIGHT authors. SIX appear on the 'first-page' (for formatting
% reasons) and the remaining two appear in the \additionalauthors section.
%
\author{
% You can go ahead and credit any number of authors here,
% e.g. one 'row of three' or two rows (consisting of one row of three
% and a second row of one, two or three).
%
% The command \alignauthor (no curly braces needed) should
% precede each author name, affiliation/snail-mail address and
% e-mail address. Additionally, tag each line of
% affiliation/address with \affaddr, and tag the
% e-mail address with \email.
%
% 1st. author
\alignauthor
Akshay Mittal\\
       \affaddr{Princeton University}\\
% 2nd. author
\alignauthor Wathsala Vithanage\\
       \affaddr{Princeton University}\\
% 3rd. author
\alignauthor Stephen Lin\\
       \affaddr{Princeton University}\\
\and  % use '\and' if you need 'another row' of author names
% 4th. author
\alignauthor Jennifer Guo\\
       \affaddr{Princeton University}\\
}

\maketitle
\begin{abstract}
This paper provides a sample of a \LaTeX\ document which conforms to
the formatting guidelines for ACM SIG Proceedings.
It complements the document \textit{Author's Guide to Preparing
ACM SIG Proceedings Using \LaTeX$2_\epsilon$\ and Bib\TeX}. This
source file has been written with the intention of being
compiled under \LaTeX$2_\epsilon$\ and BibTeX.

\end{abstract}

\section{Introduction}

Lorem ipsum dolor sit amet, et alienum salutatus cotidieque has, per an fugit dictas saperet. Cu nec odio denique, qui at assum omnium. Nihil maluisset vituperatoribus eu ius, insolens facilisis te eos. Enim elit appetere ne has, cu eos delectus inciderint. Nam ea lucilius efficiendi. An sumo diam dicta pri, partem animal aliquip eu vel. Mei et integre gloriatur delicatissimi, cu error denique duo. In mea utroque explicari contentiones, qui errem viris et. Mea epicuri periculis consequuntur eu. Cibo doctus est ne, ius dicta integre cu.


\section{Related Work}

\section{Threat Model}

\section{Framework Description}

\section{Implementation}

\section{Experimental Results}


\section{Conclusions}
This paragraph will end the body of this sample document.
Remember that you might still have Acknowledgments or
Appendices; brief samples of these
follow.  There is still the Bibliography to deal with; and
we will make a disclaimer about that here: with the exception
of the reference to the \LaTeX\ book, the citations in
this paper are to articles which have nothing to
do with the present subject and are used as
examples only.
%\end{document}  % This is where a 'short' article might terminate

%ACKNOWLEDGMENTS are optional
\section{Acknowledgments}
This section is optional; it is a location for you
to acknowledge grants, funding, editing assistance and
what have you.  In the present case, for example, the
authors would like to thank Gerald Murray of ACM for
his help in codifying this \textit{Author's Guide}
and the \textbf{.cls} and \textbf{.tex} files that it describes.

%
% The following two commands are all you need in the
% initial runs of your .tex file to
% produce the bibliography for the citations in your paper.
\bibliographystyle{abbrv}
\bibliography{sigproc}  % sigproc.bib is the name of the Bibliography in this case
% You must have a proper ".bib" file
%  and remember to run:
% latex bibtex latex latex
% to resolve all references
%
% ACM needs 'a single self-contained file'!
%
%APPENDICES are optional
%\balancecolumns
\appendix
%Appendix A

\section{Appendix Section}

Lorem ipsum dolor sit amet, et alienum salutatus cotidieque has, per an fugit dictas saperet. Cu nec odio denique, qui at assum omnium. Nihil maluisset vituperatoribus eu ius, insolens facilisis te eos. Enim elit appetere ne has, cu eos delectus inciderint. Nam ea lucilius efficiendi.


\subsection{Appendix Subsection}

Lorem ipsum dolor sit amet, et alienum salutatus cotidieque has, per an fugit dictas saperet. Cu nec odio denique, qui at assum omnium. Nihil maluisset vituperatoribus eu ius, insolens facilisis te eos. Enim elit appetere ne has, cu eos delectus inciderint. Nam ea lucilius efficiendi.


\balancecolumns
% That's all folks!
\end{document}
